The \glsdesc{acr:oqe} output of a disaggregation analysis corresponds to the
combination of a hazard curve and a multidimensional matrix containing the
results of the disaggregation. For a typical disaggregation calculation the
list of outputs are the following:

\begin{Verbatim}[frame=single, commandchars=\\\{\}, fontsize=\small]
user@ubuntu:~$ oq engine --lo <calc_id>
id | name
\textcolor{red}{3 | Disaggregation Outputs}
\textcolor{black}{5 | Full Report}
\textcolor{black}{6 | Realizations}
\end{Verbatim}
%\input{oqum/hazard/verbatim/output_disaggregation}

Running \texttt{-{}-export-output} to export the disaggregation results will produce individual files for each IMT, probability of exceedence and site.
In presence of a nontrivial logic tree the user can specify the realization
on which to perform the disaggregation by setting the \texttt{rlz\_index}
parameter in the \texttt{job.ini} file. If not specified, for each site
the engine will determine the realization closest to the mean hazard curve
and will use that realization to perform the disaggregation.

In the following inset we show an example of the nrml file used to represent the different disaggregation matrices (highlighted in red) produced by
\gls{acr:oqe}:

\input{oqum/hazard/verbatim/output_disaggregation_matrix}
