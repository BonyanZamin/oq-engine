This Chapter summarises the structure of the information necessary to define
the different input data to be used with the \glsdesc{acr:oqe} risk
calculators. Input data for scenario-based and probabilistic seismic damage
and risk analysis using the \glsdesc{acr:oqe} are organised into:

\begin{itemize}

  \item An exposure model file in the NRML format, as described in 
    Section~\ref{sec:exposure}.

  \item A file describing the \gls{vulnerabilitymodel}
    (Section~\ref{sec:vulnerability}) for loss calculations, or a 
  	file describing the \gls{fragilitymodel} (Section~\ref{sec:fragility})
    for damage calculations. Optionally, a file describing the
    \gls{consequencemodel} (Section~\ref{sec:consequence}) can also be
  	provided in order to calculate losses from the estimated damage
  	distributions.

  \item A general calculation configuration file.

  \item Hazard inputs. These include hazard curves for the classical
    probabilistic damage and risk calculators, ground motion fields for the
    scenario damage and risk calculators, or stochastic event sets for the
    probabilistic event based calculators. As of \glsdesc{acr:oqe21}, in
    general, there are five different ways in which hazard calculation
    parameters or results can be provided to the \glsdesc{acr:oqe} in order to
    run the subsequent risk calculations:

    \begin{itemize}

      \item Use a single configuration file for running the hazard and risk
      calculations sequentially (preferred)

      \item Use separate configuration files for running the hazard and risk
      calculations sequentially (legacy)

      \item Use a configuration file for the risk calculation along with all
      hazard outputs from a previously completed, compatible
      \glsdesc{acr:oqe} hazard calculation

      % \item Use a configuration file for the risk calculation along with a
      % specific hazard output from a previously completed, compatible
      % \glsdesc{acr:oqe} hazard calculation

      \item Use a configuration file for the risk calculation along with
      hazard input files in the OpenQuake NRML format

    \end{itemize}

\end{itemize}

The file formats for \glspl{exposuremodel}, \glspl{fragilitymodel},
\glspl{consequencemodel}, and \glspl{vulnerabilitymodel} have been described
earlier in Chapter~\ref{chap:riskinputs}. The configuration file is the primary
file that provides the \glsdesc{acr:oqe} information regarding both the
definition of the input models (e.g. exposure, site parameters, fragility,
consequence, or vulnerability models) as well as the parameters governing the
risk calculation.

Information regarding the configuration file for running hazard calculations
using the \glsdesc{acr:oqe} can be found in
Section~\ref{sec:hazard_configuration_file}. Some initial mandatory parameters
of the configuration file common to all of the risk calculators are presented
in Listing~\ref{lst:config_example}. The remaining parameters that are
specific to each risk calculator are discussed in subsequent sections.

\begin{listing}[htbp]
  \inputminted[firstline=1,firstnumber=1,fontsize=\footnotesize,frame=single,linenos,bgcolor=lightgray]{ini}{oqum/risk/verbatim/config_example.ini}
  \caption{Example minimal risk calculation configuration file (\href{https://raw.githubusercontent.com/gem/oq-engine/master/doc/manual/oqum/risk/verbatim/config_example.xml}{Download example})}
  \label{lst:config_example}
\end{listing}

\begin{itemize}

  \item \Verb+description+: a parameter that can be used to include some
  information about the type of calculations that are going to be performed.

  \item \Verb+calculation_mode+: this parameter specifies the type of
  calculation to be run. Valid options for the \Verb+calculation_mode+ for
  the risk calculators are: \Verb+scenario_damage+, \Verb+scenario_risk+,
  \Verb+classical_damage+, \Verb+classical_risk+, \Verb+event_based_risk+,
  and \Verb+classical_bcr+.

  \item \Verb+exposure_file+: this parameter is used to specify the path to
  the \gls{exposuremodel} file. Typically this is the path to the xml file
  containing the exposure, or the xml file containing the metadata sections
  for the case where the assets are listed in one or more csv files. 
  For particularly large exposure models, it may be more convenient to provide 
  the path to a single compressed zip file that contains the exposure xml file
  and the exposure csv files (if any).

\end{itemize}

Depending on the type of risk calculation, other parameters besides the
aforementioned ones may need to be provided. We illustrate in the following
sections different examples of the configuration file for the different risk
calculators.


\section{Scenario Damage Calculator}
\label{sec:config_scenario_damage}
\input{oqum/risk/02_config_scenario_damage}

\section{Scenario Risk Calculator}
\label{sec:config_scenario_risk}
In order to run this calculator, the parameter \Verb+calculation_mode+ needs
to be set to \Verb+scenario_risk+. 

Most of the job configuration parameters required for running a scenario risk
calculation are the same as those described in the previous section for the
scenario damage calculator. The remaining parameters specific to the scenario
risk calculator are illustrated through the examples below.


\paragraph{Example 1}

This example illustrates a scenario risk calculation which uses a single
configuration file to first compute the ground motion fields for the given
rupture model and then calculate loss statistics for structural losses
and nonstructural losses, based on the ground
motion fields. The job configuration file required for running this scenario
risk calculation is shown in Listing~\ref{lst:config_scenario_risk_combined}.

\begin{listing}[htbp]
  \inputminted[firstline=1,firstnumber=1,fontsize=\footnotesize,frame=single,linenos,bgcolor=lightgray,label=job.ini]{ini}{oqum/risk/verbatim/config_scenario_risk_combined.ini}
  \caption{Example combined configuration file for a scenario risk calculation (\href{https://raw.githubusercontent.com/gem/oq-engine/master/oqum/risk/verbatim/config_scenario_risk_combined.ini}{Download example})}
  \label{lst:config_scenario_risk_combined}
\end{listing}

Whereas a scenario damage calculation requires one or more fragility and/or
consequence models, a scenario risk calculation requires the user to specify
one or more vulnerability model files. Note that one or more of the following
parameters can be used in the same job configuration file to provide the
corresponding vulnerability model files:

\begin{itemize}

  \item \Verb+structural_vulnerability_file+: this parameter is used to
    specify the path to the structural \gls{vulnerabilitymodel} file

  \item \Verb+nonstructural_vulnerability_file+: this parameter is used to
    specify the path to the nonstructural\gls{vulnerabilitymodel} file

  \item \Verb+contents_vulnerability_file+: this parameter is used to
    specify the path to the contents \gls{vulnerabilitymodel} file

  \item \Verb+business_interruption_vulnerability_file+: this parameter is
    used to specify the path to the business interruption
    \gls{vulnerabilitymodel} file

  \item \Verb+occupants_vulnerability_file+: this parameter is used to
    specify the path to the occupants \gls{vulnerabilitymodel} file

\end{itemize}

It is important that the \Verb+lossCategory+ parameter in the metadata section
for each provided vulnerability model file (``structural'', ``nonstructural'',
``contents'', ``business\_interruption'', or ``occupants'') should match the
loss type defined in the configuration file by the relevant keyword above.

The remaining new parameters introduced in this example are the following:

\begin{itemize}

  \item \Verb+master_seed+: this parameter is used to control the random
    number generator in the loss ratio sampling process. If the same
    \Verb+master_seed+ is defined at each calculation run, the same random loss
    ratios will be generated, thus allowing reproducibility of the results.

  \item \Verb+asset_correlation+: if the uncertainty in the loss ratios
    has been defined within the \gls{vulnerabilitymodel}, users can specify
    a coefficient of correlation that will be used in the Monte Carlo sampling
    process of the loss ratios, between the assets that share the same
    \gls{taxonomy}. If the \Verb+asset_correlation+ is set to one,
    the loss ratio residuals will be perfectly correlated. On the other hand,
    if this parameter is set to zero, the loss ratios will be sampled
    independently. If this parameter is not defined, the
    \glsdesc{acr:oqe} will assume zero correlation in the vulnerability. As of
    \glsdesc{acr:oqe18}, \Verb+asset_correlation+ applies only to continuous
    \glspl{vulnerabilityfunction} using the lognormal or Beta distribution; 
    it does not apply to \glspl{vulnerabilityfunction} defined using the PMF
    distribution. Although partial correlation was supported in previous
    versions of the engine, beginning from \glsdesc{acr:oqe22}, values between
    zero and one are no longer supported due to performance considerations. The
    only two values permitted are \Verb+asset_correlation = 0+ and 
    \Verb+asset_correlation = 1+.

\end{itemize}

In this case, the ground motion fields will be computed at each of the
locations of the assets in the exposure model and for each of the intensity
measure types found in the provided set of vulnerability models. The above
calculation can be run using the command line:

\begin{minted}[fontsize=\footnotesize,frame=single,bgcolor=lightgray]{shell-session}
user@ubuntu:~\$ oq engine --run job.ini
\end{minted}

After the calculation is completed, a message similar to the following will be
displayed:

\begin{minted}[fontsize=\footnotesize,frame=single,bgcolor=lightgray]{shell-session}
Calculation 2735 completed in 10 seconds. Results:
  id | name
5328 | Aggregate Asset Losses
5329 | Average Asset Losses
5330 | Aggregate Event Losses
\end{minted}

All of the different ways of running a scenario damage calculation as
illustrated through the examples of the previous section are also applicable
to the scenario risk calculator, though the examples are not repeated here.


A few additional parameters related to the event based risk calculator that
may be useful for controlling specific aspects of the calculation are listed
below:

\begin{itemize}

  \item \Verb+ignore_covs+: this parameter controls the propagation of 
    vulnerability uncertainty to losses. The vulnerability functions using 
    continuous distributions (such as the lognormal distribution or beta 
    distribution) to characterize the uncertainty in the loss ratio 
    conditional on the shaking intensity level, specify the mean loss ratios 
    and the corresponding coefficients of variation for a set of intensity 
    levels. They are used to build the so called epsilon matrix within the 
    engine, which is how loss ratios are sampled from the distribution for 
    each asset. There is clearly a performance penalty associated with the 
    propagation of uncertainty in the vulnerability to losses. The epsilon 
    matrix has to be computed and stored, and then the worker processes have 
    to read it, which involves large quantities of data transfer and memory 
    usage. Setting \Verb+ignore_covs = true+ in the job file will result in 
    the engine using just the mean loss ratio conditioned on the shaking 
    intensity and ignoring the uncertainty. This tradeoff of not propagating 
    the vulnerabilty uncertainty to the loss estimates can lead to a 
    significant boost in performance and tractability. The default value of 
    \Verb+ignore_covs+ is \Verb+false+. 

\end{itemize}


\section{Classical Probabilistic Seismic Damage Calculator}
\label{sec:config_classical_damage}
\input{oqum/risk/02_config_classical_damage}

\section{Classical Probabilistic Seismic Risk Calculator}
\label{sec:config_classical_risk}
In order to run this calculator, the parameter \Verb+calculation_mode+ needs
to be set to \Verb+classical_risk+.

Most of the job configuration parameters required for running a classical
probabilistic risk calculation are the same as those described in the previous
section for the classical probabilistic damage calculator. The remaining
parameters specific to the classical probabilistic risk calculator are
illustrated through the examples below.

\paragraph{Example 1}

This example illustrates a classical probabilistic risk calculation which uses
a single configuration file to first compute the hazard curves for the given
source model and ground motion model and then calculate loss exceedance curves
based on the hazard curves. An example job configuration file for running a
classical probabilistic risk calculation is shown in
Listing~\ref{lst:config_classical_risk_combined}.

\begin{listing}[htbp]
  \inputminted[firstline=1,firstnumber=1,fontsize=\footnotesize,frame=single,linenos,bgcolor=lightgray,label=job.ini]{ini}{oqum/risk/verbatim/config_classical_risk_combined.ini}
  \caption{Example combined configuration file for a classical probabilistic risk calculation (\href{https://raw.githubusercontent.com/gem/oq-engine/master/doc/manual/oqum/risk/verbatim/config_classical_risk_combined.ini}{Download example})}
  \label{lst:config_classical_risk_combined}
\end{listing}

Apart from the calculation mode, the only difference with the example job
configuration file shown in Example~1 of
Section~\ref{sec:config_classical_damage} is the use of a vulnerability model
instead of a fragility model.

As with the Scenario Risk calculator, it is possible to specify one or more
\gls{vulnerabilitymodel} files in the same job configuration file, using the
parameters:

\begin{itemize}

  \item \Verb+structural_vulnerability_file+,

  \item \Verb+nonstructural_vulnerability_file+,

  \item \Verb+contents_vulnerability_file+,

  \item \Verb+business_interruption_vulnerability_file+, and/or

  \item \Verb+occupants_vulnerability_file+

\end{itemize}

It is important that the
\Verb+lossCategory+ parameter in the metadata section for each provided
vulnerability model file (``structural'', ``nonstructural'', ``contents'',
``business\_interruption'', or ``occupants'') should match the loss type
defined in the configuration file by the relevant keyword above.

In this case, the hazard curves will be computed at each of the locations of
the \glspl{asset} in the \gls{exposuremodel}, for each of the intensity
measure types found in the provided set of \glspl{vulnerabilitymodel}. The
above calculation can be run using the command line:

\begin{minted}[fontsize=\footnotesize,frame=single,bgcolor=lightgray]{shell-session}
user@ubuntu:~\$ oq engine --run job.ini
\end{minted}

After the calculation is completed, a message similar to the following will be
displayed:

\begin{minted}[fontsize=\footnotesize,frame=single,bgcolor=lightgray]{shell-session}
Calculation 2749 completed in 24 seconds. Results:
  id | name
3980 | Asset Loss Curves Statistics
3981 | Asset Loss Maps Statistics
3983 | Average Asset Loss Statistics
\end{minted}


\paragraph{Example 2}

This example illustrates a classical probabilistic risk calculation which uses
separate configuration files for the hazard and risk parts of a classical
probabilistic risk assessment. The first configuration file shown in
Listing~\ref{lst:config_classical_risk_hazard} contains input models and
parameters required for the computation of the hazard curves. The second
configuration file shown in Listing~\ref{lst:config_classical_risk} contains
input models and parameters required for the calculation of the loss
exceedance curves and probabilistic loss maps for a portfolio of \glspl{asset}
based on the hazard curves and \glspl{vulnerabilitymodel}.

\begin{listing}[htbp]
  \inputminted[firstline=1,firstnumber=1,fontsize=\footnotesize,frame=single,linenos,bgcolor=lightgray,label=job\_hazard.ini]{ini}{oqum/risk/verbatim/config_classical_hazard.ini}
  \caption{Example hazard configuration file for a classical probabilistic risk calculation (\href{https://raw.githubusercontent.com/gem/oq-engine/master/doc/manual/oqum/risk/verbatim/config_classical_hazard.ini}{Download example})}
  \label{lst:config_classical_risk_hazard}
\end{listing}

\begin{listing}[htbp]
  \inputminted[firstline=1,firstnumber=1,fontsize=\footnotesize,frame=single,linenos,bgcolor=lightgray,label=job_risk.ini]{ini}{oqum/risk/verbatim/config_classical_risk.ini}
  \caption{Example risk configuration file for a classical probabilistic risk calculation (\href{https://raw.githubusercontent.com/gem/oq-engine/master/doc/manual/oqum/risk/verbatim/config_classical_risk.ini}{Download example})}
  \label{lst:config_classical_risk}
\end{listing}

Now, the above calculations described by the two configuration files
``job\_hazard.ini'' and ``job\_risk.ini'' can be run sequentially or
separately, as illustrated in Example~2 in
Section~\ref{sec:config_scenario_damage}. The new parameters introduced in the
above risk configuration file example
(Listing~\ref{lst:config_classical_risk}) are described below:

\begin{itemize}

	\item \Verb+lrem_steps_per_interval+: this parameter controls the number of
	  intermediate values between consecutive loss ratios (as defined in the 
	  \gls{vulnerabilitymodel}) that are considered in the risk calculations.
	  A larger number of loss ratios than those defined in each
	  \gls{vulnerabilityfunction} should be considered, in order to better
	  account for the uncertainty in the loss ratio distribution. If this
	  parameter is not defined in the configuration file, the \glsdesc{acr:oqe}
	  assumes the \Verb+lrem_steps_per_interval+ to be equal to 5. More details
	  are provided in the OpenQuake Book (Risk).

	\item \Verb+quantiles+: this parameter can be used to request
	  the computation of quantile loss curves for computations involving
	  non-trivial logic trees. The quantiles for which the loss curves should
	  be computed must be provided as a comma separated list. If this parameter
	  is not included in the configuration file, quantile loss curves will not
	  be computed. 

	\item \Verb+conditional_loss_poes+: this parameter can be used to request
	  the computation of probabilistic loss maps, which give the loss levels
	  exceeded at the specified probabilities of exceedance over the time
	  period specified by \Verb+risk_investigation_time+. The probabilities of
	  exceedance for which the loss maps should be computed must be provided as
	  a comma separated list. If this parameter is not included in the
	  configuration file, probabilistic loss maps will not be computed.

\end{itemize}


\section{Stochastic Event Based Seismic Damage Calculator}
\label{sec:config_event_based_damage}
The parameter \Verb+calculation_mode+ needs to be set to
\Verb+event_based_damage+ in order to use this calculator.

Most of the job configuration parameters required for running a stochastic
event based damage calculation are the same as those described in the previous
sections for the scenario damage calculator and the classical probabilistic damage
calculator. The remaining parameters specific to the stochastic event based
damage calculator are illustrated through the example below.


\paragraph{Example 1}

This example illustrates a stochastic event based damage calculation which uses
a single configuration file to first compute the \glspl{acr:ses} and
\glspl{acr:gmf} for the given source model and ground motion model, and then
calculate event loss tables, loss exceedance curves and probabilistic
loss maps for structural losses, nonstructural losses and occupants,
based on the \glspl{acr:gmf}. The job configuration file required for
running this stochastic event based damage calculation is shown in
Listing~\ref{lst:config_event_based_damage}.

\begin{listing}[htbp]
  \inputminted[firstline=1,firstnumber=1,fontsize=\scriptsize
  ,frame=single,bgcolor=lightgray,linenos,label=job.ini]{ini}{oqum/risk/verbatim/config_event_based_damage.ini}
  \caption{Example configuration file for running a stochastic event based damage calculation (\href{https://raw.githubusercontent.com/gem/oq-engine/master/doc/manual/oqum/risk/verbatim/config_event_based_damage.ini}{Download example})}
  \label{lst:config_event_based_damage}
\end{listing}

Similar to that the procedure described for the Scenario Damage calculator, a
Monte Carlo sampling process is also employed in this calculator to take into
account the uncertainty in the conditional loss ratio at a particular
intensity level. Hence, the parameters \Verb+asset_correlation+ and
\Verb+master_seed+ may be defined as previously described for the Scenario
Damage calculator in Section~\ref{sec:config_scenario_damage}. The parameter
``risk\_investigation\_time'' specifies the time period for which the average
damage values will be calculated, similar to the
Classical Probabilistic Damage calculator. If this parameter is not provided in
the risk job configuration file, the time period used is the same as that
specifed in the hazard calculation using the parameter ``investigation\_time''.

The new parameters introduced in this example are described below:

\begin{itemize}

  \item \Verb+minimum_intensity+: this optional parameter specifies the minimum
    intensity levels for each of the intensity measure types in the risk model.
    Ground motion fields where each ground motion value is less than the 
    specified minimum threshold are discarded. This helps speed up calculations
    and reduce memory consumption by considering only those ground motion fields
    that are likely to contribute to losses. It is also possible to set the same
    threshold value for all intensity measure types by simply providing a single
    value to this parameter. For instance: ``minimum\_intensity = 0.05'' would
    set the threshold to 0.05 g for all intensity measure types in the risk 
    calculation.
    If this parameter is not set, the \glsdesc{acr:oqe} extracts the minimum
    thresholds for each intensity measure type from the vulnerability
    models provided, picking the lowest intensity value for which a mean loss
    ratio is provided.

  \item \Verb+return_periods+: this parameter specifies the list of return
    periods (in years) for computing the asset / aggregate damage curves.
    If this parameter is not set, the \glsdesc{acr:oqe} uses a default set of
    return periods for computing the loss curves. The default return periods
    used are from the list: [5, 10, 25, 50, 100, 250, 500, 1000, ...], with 
    its upper bound limited by \Verb+(ses_per_logic_tree_path × investigation_time)+

    \begin{equation*}
    \begin{split}
    average\_damages & = sum(event\_damages) \\
                 & \div (hazard\_investigation\_time \times ses\_per\_logic\_tree\_path) \\
                 & \times risk\_investigation\_time
    \end{split}
    \end{equation*}

\end{itemize}

The above calculation can be run using the command line:

\begin{minted}[fontsize=\footnotesize,frame=single,bgcolor=lightgray]{shell-session}
user@ubuntu:~\$ oq engine --run job.ini
\end{minted}

Computation of the damage curves, and average damages for each
individual \gls{asset} in the \gls{exposuremodel} can be resource intensive,
and thus these outputs are not generated by default.

\section{Stochastic Event Based Seismic Risk Calculator}
\label{sec:config_event_based_risk}
The parameter \Verb+calculation_mode+ needs to be set to
\Verb+event_based_risk+ in order to use this calculator.

Most of the job configuration parameters required for running a stochastic
event based risk calculation are the same as those described in the previous
sections for the scenario risk calculator and the classical probabilistic risk
calculator. The remaining parameters specific to the stochastic event based
risk calculator are illustrated through the example below.


\paragraph{Example 1}

This example illustrates a stochastic event based risk calculation which uses
a single configuration file to first compute the \glspl{acr:ses} and
\glspl{acr:gmf} for the given source model and ground motion model, and then
calculate event loss tables, loss exceedance curves and probabilistic
loss maps for structural losses, nonstructural losses and occupants,
based on the \glspl{acr:gmf}. The job configuration file required for
running this stochastic event based risk calculation is shown in
Listing~\ref{lst:config_event_based_risk_combined}.

\begin{listing}[htbp]
  \inputminted[firstline=1,firstnumber=1,fontsize=\scriptsize
  ,frame=single,bgcolor=lightgray,linenos,label=job.ini]{ini}{oqum/risk/verbatim/config_event_based_risk_combined.ini}
  \caption{Example combined configuration file for running a stochastic event based risk calculation (\href{https://raw.githubusercontent.com/gem/oq-engine/master/doc/manual/oqum/risk/verbatim/config_event_based_risk_combined.ini}{Download example})}
  \label{lst:config_event_based_risk_combined}
\end{listing}

Similar to that the procedure described for the Scenario Risk calculator, a
Monte Carlo sampling process is also employed in this calculator to take into
account the uncertainty in the conditional loss ratio at a particular
intensity level. Hence, the parameters \Verb+asset_correlation+ and
\Verb+master_seed+ may be defined as previously described for the Scenario
Risk calculator in Section~\ref{sec:config_scenario_risk}. The parameter
``risk\_investigation\_time'' specifies the time period for which the event
loss tables and loss exceedance curves will be calculated, similar to the
Classical Probabilistic Risk calculator. If this parameter is not provided in
the risk job configuration file, the time period used is the same as that
specifed in the hazard calculation using the parameter ``investigation\_time''.

The new parameters introduced in this example are described below:

\begin{itemize}

  \item \Verb+minimum_intensity+: this optional parameter specifies the minimum
    intensity levels for each of the intensity measure types in the risk model.
    Ground motion fields where each ground motion value is less than the 
    specified minimum threshold are discarded. This helps speed up calculations
    and reduce memory consumption by considering only those ground motion fields
    that are likely to contribute to losses. It is also possible to set the same
    threshold value for all intensity measure types by simply providing a single
    value to this parameter. For instance: ``minimum\_intensity = 0.05'' would
    set the threshold to 0.05 g for all intensity measure types in the risk 
    calculation.
    If this parameter is not set, the \glsdesc{acr:oqe} extracts the minimum
    thresholds for each intensity measure type from the vulnerability
    models provided, picking the lowest intensity value for which a mean loss
    ratio is provided.

  \item \Verb+return_periods+: this parameter specifies the list of return
    periods (in years) for computing the aggregate loss curve.
    If this parameter is not set, the \glsdesc{acr:oqe} uses a default set of
    return periods for computing the loss curves. The default return periods
    used are from the list: [5, 10, 25, 50, 100, 250, 500, 1000, ...], with 
    its upper bound limited by \Verb+(ses_per_logic_tree_path × investigation_time)+

  \item \Verb+avg_losses+: this boolean parameter specifies whether the average
    asset losses over the time period ``risk\_investigation\_time'' should be
    computed. The default value of this parameter is \Verb+true+.

    \begin{equation*}
    \begin{split}
    average\_loss & = sum(event\_losses) \\
                 & \div (hazard\_investigation\_time \times ses\_per\_logic\_tree\_path) \\
                 & \times risk\_investigation\_time
    \end{split}
    \end{equation*}

\end{itemize}

The above calculation can be run using the command line:

\begin{minted}[fontsize=\footnotesize,frame=single,bgcolor=lightgray]{shell-session}
user@ubuntu:~\$ oq engine --run job.ini
\end{minted}

Computation of the loss tables, loss curves, and average losses for each
individual \gls{asset} in the \gls{exposuremodel} can be resource intensive,
and thus these outputs are not generated by default, unless instructed to by
using the parameters described above.


Starting from \gls{acr:oqe28}, users may also begin a stochastic event based
risk calculation by providing a precomputed set of \glspl{acr:gmf} to the
\gls{acr:oqe}. The following example describes the procedure for this
approach.

\paragraph{Example 2}

This example illustrates a stochastic event based risk calculation which uses
a file listing a precomputed set of \glspl{acr:gmf}. These \glspl{acr:gmf} can
be computed using the \glsdesc{acr:oqe} or some other software. The
\glspl{acr:gmf} must be provided in either the \gls{acr:nrml} schema or the
csv format as presented in Section~\ref{subsec:output_event_based_psha}.
Listing~\ref{lst:output_gmf_xml} shows an example of a \glspl{acr:gmf} file in
the \gls{acr:nrml} schema and Table~\ref{output:gmf_event_based} shows an
example of a \glspl{acr:gmf} file in the csv format. If the \glspl{acr:gmf}
file is provided in the csv format, an additional csv file listing the site
ids must be provided using the parameter \Verb+sites_csv+. See
Table~\ref{output:sitemesh} for an example of the sites csv file, which
provides the association between the site ids in the \glspl{acr:gmf} csv file
with their latitude and longitude coordinates.

Starting from the input \glspl{acr:gmf}, the \gls{acr:oqe} can calculate event
loss tables, loss exceedance curves and probabilistic loss maps for structural
losses, nonstructural losses and occupants. The
job configuration file required for running this stochastic event based risk
calculation starting from a precomputed set of \glspl{acr:gmf} is shown in
Listing~\ref{lst:config_gmf_event_based_risk}.

\begin{listing}[htbp]
  \inputminted[firstline=1,firstnumber=1,fontsize=\scriptsize
  ,frame=single,bgcolor=lightgray,linenos,label=job.ini]{ini}{oqum/risk/verbatim/config_gmf_event_based_risk.ini}
  \caption{Example combined configuration file for running a stochastic event based risk calculation starting from a precomputed set of ground motion fields (\href{https://raw.githubusercontent.com/gem/oq-engine/master/doc/manual/oqum/risk/verbatim/config_gmf_event_based_risk.ini}{Download example})}
  \label{lst:config_gmf_event_based_risk}
\end{listing}


\section{Retrofit Benefit-Cost Ratio Calculator}
\label{sec:config_benefit_cost}
\input{oqum/risk/02_config_benefit_cost}

\section{Exporting Risk Results}
\label{sec:risk_export}
To obtain a list of all risk calculations that have been previously run
(successfully or unsuccessfully), or that are currently running, the following
command can be employed:

\begin{minted}[fontsize=\footnotesize,frame=single,bgcolor=lightgray]{shell-session}
user@ubuntu:~\$ oq engine --list-risk-calculations
\end{minted}

or simply:

\begin{minted}[fontsize=\footnotesize,frame=single,bgcolor=lightgray]{shell-session}
user@ubuntu:~\$ oq engine --lrc
\end{minted}

Which will display a list of risk calculations as presented below.

\begin{minted}[fontsize=\footnotesize,frame=single,bgcolor=lightgray]{shell-session}
job_id |     status |          start_time |     description
     1 |   complete | 2015-12-02 08:50:30 | Scenario damage example
     2 |     failed | 2015-12-03 09:56:17 | Scenario risk example
     3 |   complete | 2015-12-04 10:45:32 | Scenario risk example
     4 |   complete | 2015-12-04 10:48:33 | Classical risk example
     5 |   complete | 2020-07-09 13:47:45 | Event based risk aggregation example     
\end{minted}

Then, in order to display a list of the risk outputs from a given job which has
completed successfully, the following command can be used:

\begin{minted}[fontsize=\footnotesize,frame=single,bgcolor=lightgray]{shell-session}
user@ubuntu:~\$ oq engine --list-outputs <risk_calculation_id>
\end{minted}

or simply:

\begin{minted}[fontsize=\footnotesize,frame=single,bgcolor=lightgray]{shell-session}
user@ubuntu:~\$ oq engine --lo <risk_calculation_id>
\end{minted}

which will display a list of outputs for the calculation requested, as
presented below:

\begin{minted}[fontsize=\footnotesize,frame=single,bgcolor=lightgray]{shell-session}
Calculation 5 results:
  id | name
  11 | Aggregate Event Losses
   1 | Aggregate Loss Curves
   2 | Aggregate Loss Curves Statistics
   3 | Aggregate Losses
   4 | Aggregate Losses Statistics
   5 | Average Asset Losses Statistics
  13 | Earthquake Ruptures
   6 | Events
   7 | Full Report
  10 | Input Files
  12 | Realizations
  14 | Source Loss Table
  15 | Total Loss Curves
  16 | Total Loss Curves Statistics
  17 | Total Losses
  18 | Total Losses Statistics
\end{minted}

Then, in order to export all of the risk calculation outputs in the
default file format (csv for most outputs), the following command can be used:

\begin{minted}[fontsize=\footnotesize,frame=single,bgcolor=lightgray]{shell-session}
user@ubuntu:~\$ oq engine --export-outputs <risk_calculation_id> <output_directory>
\end{minted}

or simply:

\begin{minted}[fontsize=\footnotesize,frame=single,bgcolor=lightgray]{shell-session}
user@ubuntu:~\$ oq engine --eos <risk_calculation_id> <output_directory>
\end{minted}

If, instead of exporting all of the outputs from a particular calculation,
only particular output files need to be exported, this can be achieved by
using the \Verb+--export-output+ option and providing the id of the required
output:

\begin{minted}[fontsize=\footnotesize,frame=single,bgcolor=lightgray]{shell-session}
user@ubuntu:~\$ oq engine --export-output <risk_output_id> <output_directory>
\end{minted}

or simply:

\begin{minted}[fontsize=\footnotesize,frame=single,bgcolor=lightgray]{shell-session}
user@ubuntu:~\$ oq engine --eo <risk_output_id> <output_directory>
\end{minted}


\cleardoublepage